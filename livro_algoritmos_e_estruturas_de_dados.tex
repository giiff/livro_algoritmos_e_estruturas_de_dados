%!TEX program = pdflatex
%----------------------------------------------------------------------------------------
% TEMPLATE INFORMATON
%----------------------------------------------------------------------------------------
% This template was adapted from "The Legrand Orange Book Version 1.4 (12/4/14) [http://www.LaTeXTemplates.com] Mathias Legrand (legrand.mathias@gmail.com)" License: CC BY-NC-SA 3.0 (http://creativecommons.org/licenses/by-nc-sa/3.0/)
%----------------------------------------------------------------------------------------

%----------------------------------------------------------------------------------------
% COMPILACAO
%----------------------------------------------------------------------------------------
% 0) There is a "compilar.sh"
% OR
% 1) pdflatex livro_algoritmos_e_estruturas_de_dados
% 2) makeindex livro_algoritmos_e_estruturas_de_dados.idx -s StyleInd.ist
% 3) biber livro_algoritmos_e_estruturas_de_dados
% 4) pdflatex livro_algoritmos_e_estruturas_de_dados
% 5) pdflatex livro_algoritmos_e_estruturas_de_dados

%----------------------------------------------------------------------------------------
%	PACKAGES AND OTHER DOCUMENT CONFIGURATIONS
%----------------------------------------------------------------------------------------
\documentclass[11pt,fleqn]{book} % Default font size and left-justified equations

%encoding
%--------------------------------------
\usepackage[T1]{fontenc}
%--------------------------------------
\usepackage[utf8]{inputenc}
\hyphenation{Mi-nis-té-ri-o}
\usepackage[top=3cm,bottom=3cm,left=3.2cm,right=3.2cm,headsep=10pt,a4paper]{geometry} % Page margins
\usepackage[svgnames]{xcolor} % Required for specifying colors by name
\definecolor{blue}{rgb}{0.0, 0.18, 0.39}
% Font Settings
\usepackage{avant} % Use the Avantgarde font for headings
%\usepackage{times} % Use the Times font for headings
\usepackage{mathptmx} % Use the Adobe Times Roman as the default text font together with math symbols from the Sym­bol, Chancery and Com­puter Modern fonts
\usepackage{microtype} % Slightly tweak font spacing for aesthetics
% Index
\usepackage{calc} % For simpler calculation - used for spacing the index letter headings correctly
\usepackage{makeidx} % Required to make an index
\makeindex % Tells LaTeX to create the files required for indexing
\usepackage{verbatim}
% Others
\usepackage[colorinlistoftodos,prependcaption,textsize=tiny,linecolor=red,backgroundcolor=red!25,bordercolor=red]{todonotes}
\usepackage{epigraph}
\renewcommand{\textflush}{flushepinormal}
\setlength{\epigraphwidth}{0.8\textwidth}

\usepackage{nameref}
\usepackage{booktabs}
\usepackage{graphicx}
\usepackage{float}
\usepackage{multirow}

%CODE
\usepackage{courier}
\usepackage{listings}
\lstset{
	numbers=left,                   % where to put the line-numbers
	stepnumber=1,                   % the step between two line-numbers. 	
	frame=tb,                       % draw a frame at the top and bottom of the code block
	tabsize=4,                      % tab space width
	showstringspaces=false,         % don't mark spaces in strings
	numbers=left,                   % display line numbers on the left
	commentstyle=\color{Gray},      % comment color
	keywordstyle=\color{DarkBlue},  % keyword color
	stringstyle=\color{Maroon},     % string color
	breaklines=true,                % break lines when it's needed
	basicstyle=\ttfamily\scriptsize % font family and size (need package courier)
}
\renewcommand{\lstlistingname}{Programa}

% Bibliography
%\usepackage[backend=biber,style=authoryear,autocite=inline, citestyle=authoryear]{biblatex}
\usepackage[style=abnt]{biblatex}
\addbibresource{bibliography.bib} % BibTeX bibliography file
\defbibheading{bibempty}{}
\renewcommand*{\nameyeardelim}{\addcomma\space}

\newcommand{\VER}[1]{\begingroup\color{red}#1\endgroup}

%----------------------------------------------------------------------------------------

\input{structure} % Insert the commands.tex file which contains the majority of the structure behind the template

\begin{document}

\let\cleardoublepage\clearpage

\renewcommand{\chaptername}{Capítulo}
\renewcommand{\figurename}{Fig.}

%----------------------------------------------------------------------------------------
%	TITLE PAGE
%----------------------------------------------------------------------------------------
\begingroup
	\thispagestyle{empty}
	
	\AddToShipoutPicture*{\put(0,0){\includegraphics[scale=1]{capa}}} % Image background
	
	%\AddToShipoutPicture*{\put(116,650){\includegraphics[scale=.75]{brasao.png}}} % Image background
	
	%\AddToShipoutPicture*{\put(244,200){\includegraphics[scale=0.2]{ifgvertical}}} % Image background
	\AddToShipoutPicture*{\put(244,650){\includegraphics[scale=0.2]{ifgvertical}}} % Image background
	
	\vspace*{4.5cm}
	
	\centering
	\par
	\fontsize{30}{30}
	\selectfont
	Algoritmos e Estruturas de Dados \\
	\vspace*{1.5cm}
	\par
	\fontsize{16}{16}
	\selectfont
	Tecnólogo em Análise e Desenvolvimento de Sistemas\\
	\vspace*{1.5cm}
	Waldeyr Mendes Cordeiro da Silva\\
	\vspace*{10cm}
	\par
	{\Huge 2019}
	\par
\endgroup
\pagebreak

%----------------------------------------------------------------------------------------
%	PEOPLE PAGE
%----------------------------------------------------------------------------------------
\chapterimage{banner3} % Chapter heading image
\par
\section*{Material didático para Algoritmos e Estruturas de Dados}

Versão 0.1

\section*{Waldeyr Mendes Cordeiro da Silva}\label{WaldeyrMendes}
\begin{itemize}
	\item Formação:
	\begin{itemize}
		\item Bacharelado em Sistemas de Informação
		\item Licenciatura em Ciências Biológicas
		\item Complementação Pedagógica em Matemática
		\item Especialização em Engenharia de Software
		\item Especialização em Segurança da Informação
		\item Mestrado em Informática
		\item Doutorado em Ciências Biológicas (Bioinformática)
	\end{itemize}
	\item \includegraphics[scale=.03]{Pictures/lattes}~\href{http://lattes.cnpq.br/2391349697609978}{Lattes: http://lattes.cnpq.br/2391349697609978}
	\item \includegraphics[scale=.15]{Pictures/orcid}~\href{https://orcid.org/0000-0002-8660-6331}{ORCID: https://orcid.org/0000-0002-8660-6331}
\end{itemize}

\chapterimage{banner3} % Chapter heading image
\renewcommand\contentsname{Sumário}
\tableofcontents

%----------------------------------------------------------------------------------------
%	CHAPTER
%----------------------------------------------------------------------------------------
\chapterimage{01.jpg} % Chapter heading image
\chapter{Prefácio}\label{prefacio}
\vspace{6em}
\begin{flushright}
	\textit{\textcolor{white}{Um bonita citação...}}
\end{flushright}
\vspace{12em}

\todo[inline]{Em construção...}

Este livro destina-se aos acadêmicos e demais interessados em iniciar os estudos em algoritmos e estruturas de dados.

%----------------------------------------------------------------------------------------
%	CHAPTER
%----------------------------------------------------------------------------------------
\chapterimage{02.jpg} % Chapter heading image
\chapter{Introdução}\label{introducao}
\vspace{6em}
\begin{flushright}
	\textit{\textcolor{white}{Um bonita citação...}}
\end{flushright}
\vspace{12em}


Um algoritmo é um procedimento computacional bem definido que processa um valor ou um conjunto de dados (entrada) e produz algum valor ou conjunto de dados (saída)~\cite{cormen2009}.
Os algoritmos existem há muito tempo, mas estão presentes na sociedade moderna de uma forma nunca experimentada na história da humanidade.
Quase tudo que se produz, sejam produtos ou serviços, tem alguma influência de algoritmos.
O comércio eletrônico utiliza tanto algoritmos clássicos como ordenações, quanto algoritmos modernos de recomendação de produtos baseado no histórico de visitas.
A segurança das senhas em qualquer sistema bancário ou \textit{Web} é garantida por algoritmos de criptografia.
Imagens de satélite, dados genômicos, reconhecimento de faces, previsão do tempo, quase tudo que se possa imaginar atualmente é influenciado direta oou indiretamente por algum algoritmo.

O estudo de algoritmos é uma demanda crescente frente aos novos desafios trazidos pelo grande volume de dados que as tecnologias modernas proporcionaram.
A análise de algortimos é uma área com questões importantes em aberto, como é o caso dos \textit{Millennium Prize Problems}, com sete problemas matemático-computacionais em aberto e cujo prêmio é de 1 milhão de dólares por cada solução.

O propósito da análise de um algortimo é prever seu comportamento quanto ao tempo de execução ou ao espaço em memória que irá ocupar mesmo antes de ser executado em um computador específico.
Porém, muitos fatores, como o tamanho e a variedade dos dos dados de entrada, influenciam o algoritmo.
Portanto, a análise de algoritmos provê uma aproximação, o que em muitos casos é bastante significante.

Tipos abstratos de dados são conjuntos de valores sobre os quais é possível aplicar funções de forma homogênea através de um algortimo.
Funções e valores em conjunto, consituem um modelo matemático que pode ser empregado em problemas do mundo real~\cite{ascencio2010}.
Os algortimos são projetados em função de um tipo abstrato de dados.
A representação computacional de um tipo abstrato de dados com seus tipos e operações permitidas pode ser entendida como uma \textbf{estrutura de dados}.
Uma estrutura de dados é um meio para aramzenar e processar dados com vistas à sua organização, acesso e modificações~\cite{cormen2009}.

\section{Análise de Algoritmos}\label{sec_analise}

\subsection{Elementos de Notação Assintótica}




%----------------------------------------------------------------------------------------
%	CHAPTER
%----------------------------------------------------------------------------------------
%------------------------------------------------
\chapterimage{04.jpg} % Chapter heading image
\chapter{Programação}\label{programacao}
\vspace{6em}
\begin{flushright}
	\textit{\textcolor{white}{Um bonita citação...}}
\end{flushright}
\vspace{12em}



\newpage
\section{Primeiros Passos em Programação}\label{disc:primeirospassos}

\lstinputlisting[language=C, caption={Meu primeiro programa em C.}]{Code/Basics/prog001.c}\label{prog001}



%----------------------------------------------------------------------------------------
%	CHAPTER X
%----------------------------------------------------------------------------------------
%------------------------------------------------
\chapterimage{05.jpg} % Chapter heading image
\chapter{Estruturas de Dados}\label{estrutura}
\vspace{6em}
\begin{flushright}
	\textit{\textcolor{white}{Um bonita citação...}}
\end{flushright}
\vspace{12em}

\newpage
\section{Estruturas de Dados Homogêneas e Heterogêneas}\label{tipos}

\subsection*{Vetores}

Primeiro, uma revisão sobre vetores.
Em C uma variável que represente um vetor é um ponteiro.
Portanto, quando um vetor é parâmetro para uma função o que é passado é a sua referência, ou seja, o endereço base do vetor.
Vetores podem ser bi-, tri-, ou multi-domensionais, mas abrigam o mesmo tipo de dados.
O programa~\ref{rev001} mostra o vetor sendo passado para uma função que retorna o valor do meio do vetor.

\lstinputlisting[language=C, caption={Em C, vetores são passados para funções por referência.}]{Code/Revisao/rev001.c}\label{rev001}

\lstinputlisting[language=C, caption={Um vetor de duas dimensões mostrando apenas valores da diagonal principal.}]{Code/Revisao/rev003.c}\label{rev003}

\subsection*{Strings em C}
Em C, uma \textit{string} é um vetor de caracteres e cada \textit{string} termina com um caracter \textit{NULL}.
Uma constante \textit{string} é definida dentro de aspas em que o caractere \textit{NULL} é automaticamente incluído.
Por exemplo, a string "IFG" é um vetor de 4 elementos.
O programa~\ref{rev002} mostra um exemplo de \textit{string} em C. 
\lstinputlisting[language=C, caption={Em C, \textit{strings} são vetores de caracteres..}]{Code/Revisao/rev002.c}\label{rev002}

\newpage
\section{Algoritmos de Ordenação}\label{ordenacao}

\subsection*{Bubble Sort}
O algoritmo da bolha (\textit{Bubble Sort}) é um algortimo de ordenação onde cada elemento de uma posição $i$ é comparado com o elemento de posição $i+1$, os quais trocam de posição, se for o caso, para atender à ordenação procurada (crescente ou decrescente).
O  programa~\ref{BubbleSort} apresenta uma implmentação em C do algoritmo \textit{Bubble Sort} enquanto o Figura~\ref{diaBubleSort} ilustra seu funcionamento para um vetor de tamanho 4.
\lstinputlisting[language=C, caption={Implementação em C do algoritmo de ordenação BubbleSort.}]{Code/Ordenacao/BubbleSort.c}\label{BubbleSort}
\begin{figure}
	\centering
	\includegraphics[width=.75\textwidth]{Pictures/BubbleSort}
	\caption[Bubble Sort]{Algoritmo \textit{Bubble Sort} para um vetor de tamanho 4.}
	\label{diaBubleSort}
\end{figure}

\subsection*{Exercícios}
\begin{enumerate}
	\item Atualizar a implementação do \textit{Bubble Sort} para executar as seguintes tarefas:
	\begin{itemize}
		\item tamanho do vetor = 100000;
		\item criar variáveis contadoras para quantidade de comparações e quantidade de trocas;
		\item executar o algoritmo em ordenação crescente para o vetor com valores já ordenados, com valores aleatórios e com valores decrescentes;
	\end{itemize} 
	\item criar gráficos para as três execuções comparando quantidade de comparações e quantidade de trocas e o tempo de execução;
\end{enumerate} 

\section{Listas}\label{listas}

\section{Pilhas}\label{pilhas}

\section{Filas}\label{filas}

\section{Tabelas Hashing}\label{hashing}

\section{Árvores}\label{arvores}

\section{Grafos}\label{grafos}

\subsection{Busca em Grafos}

%----------------------------------------------------------------------------------------
%	CHAPTER X
%----------------------------------------------------------------------------------------
%------------------------------------------------
\chapterimage{06.jpg} % Chapter heading image
\chapter{Aplicações}\label{aplicacoes}
\vspace{6em}
\begin{flushright}
	\textit{\textcolor{white}{Um bonita citação...}}
\end{flushright}
\vspace{12em}



% ----------------------------------------------------------------------------------------
% 	BIBLIOGRAPHY
% ----------------------------------------------------------------------------------------
%----------------------------------------------------------------------------------------
%	CHAPTER X
%----------------------------------------------------------------------------------------
%------------------------------------------------
\chapterimage{07.jpg} % Chapter heading image
%\chapter*{Referências Bibliográficas}
%\bibliography{bibliography}
%\renewcommand\bibname{Referências Bibliográficas}

\chapter*{Referências Bibliográficas}\label{referencias}
\vspace{6em}
\begin{flushright}
	\textit{\textcolor{white}{Um bonita citação...}}
\end{flushright}
\vspace{12em}
%\addcontentsline{toc}{chapter}{\textcolor{verde}{Bibliography}}
%\section{Books}
%\addcontentsline{toc}{section}{Books}
%\printbibliography[heading=bibempty,type=book]
%\section{Articles}
%\addcontentsline{toc}{section}{Articles}
%\printbibliography[heading=bibempty,type=article]
\printbibliography[heading=bibempty]


%----------------------------------------------------------------------------------------

%----------------------------------------------------------------------------------------
%	INDEX
%----------------------------------------------------------------------------------------
%
%\cleardoublepage
%\phantomsection
%\setlength{\columnsep}{0.75cm}
%\addcontentsline{toc}{chapter}{\textcolor{verde}{Index}}
%\printindex


%----------------------------------------------------------------------------------------
%	CHAPTER X
%----------------------------------------------------------------------------------------
%------------------------------------------------
\chapterimage{08.jpg} % Chapter heading image
\chapter{Apêncice}\label{apendice}
\vspace{6em}
\begin{flushright}
	\textit{\textcolor{white}{}}
\end{flushright}
\vspace{12em}


\end{document}
